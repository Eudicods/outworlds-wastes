By default, \emph{BattleTech: Outworlds Wastes} assumes commanders are using the full rules for combined arms from \emph{BattleTech: Total Warfare} and \emph{Alpha Strike: Commander's Edition}.
League organizers may allow Battlefield Support: Strikes rules (p. 75, \emph{BMM}  and p.55, \emph{AS: CE}) to provide simplified aerospace and artillery support and off-board artillery rules for Alpha Strike (see p. 151, \emph{AS: CE}).
League organizers may also allow Battlefield Support: Assets rules (BattleTech: Mercenaries box set) to provide simplified non-'Mech units, such as combat vehicles and infantry.

\subsubsection{Battlefield Support: Strikes and Off-Board Artillery}

Each side for a scenario must agree to use Battlefield Support: Strikes or off-board artillery rules.
When using Battlefield Support: Strikes rules, each side receives the same number of Battlefield Support Points (BSP).
Consult the Battlefield Support rules in Alpha Strike: Commander's Edition or BattleTech: Mercenaries box set, as appropriate, to help determine the number of BSP and the cost of each support type.
Each side may also agree to use any unused BV from force selection to purchase additional BSP, where 1 BSP costs 20 BV.
Commanders may increase their BSP by up to half of the BSP limit.

The following additional rules apply when using Battlefield Support: Strikes or off-board artillery.

\begin{itemize}

\item The commander's force must have a unit capable of offering the support.

\begin{table}[!h]
\ifthenelse{\not \equal{\outworldsMode}{mode-web}}{\fontfamily{Montserrat-LF}}{\small}\selectfont
\centering
\newcolumntype{R}[1]{>{\raggedleft\let\newline\\\arraybackslash\hspace{0pt}}m{#1}}
\begin{tabular}{!{\Vline{1pt}} m{18em} m{20em} !{\Vline{1pt}}}
\Hline{1pt}
\rowcolor{black!30}  \bfseries{Support Type} & \bfseries{Required Unit} \\
\Hline{1pt}
Offensive Aerospace Support (BSP)   & Attack or fire-support aerospace unit    \\
Defensive Aerospace Support (BSP)   & Dogfighter or interceptor aerospace unit \\
Artillery Support (BSP)             & Corresponding artillery unit             \\
Counter-Battery Support (BSP)       & Artillery or aerospace unit              \\
Minefield Support (BSP)             & Any unit                                 \\
Off-board Artillery (AS)            & Artillery unit                           \\
Counter-Battery Fire (AS)           & Artillery or aerospace unit              \\
\Hline{1pt}
\end{tabular}
\caption*{Support Unit Requirements}
\end{table}

\item The commander must declare which off-map unit in their force is offering support during the scenario before using the off-map support rules.
The same unit must be used to provide this type of support for the rest of the scenario, unless this unit is destroyed.

\item Commanders declare what Battlefield Support they are using this turn at the start of the Weapon Attack Phase while declaring attacks.

\item The first successful use of Defensive Aerospace Support or Counter-Battery Support/Fire damages the attacking unit.
The second successful use destroys the attacking unit.

\end{itemize}

Counter-Battery Support is a new form of Battlefield Support for BattleTech that mirrors the Counter-Battery Fire rule from Alpha Strike.

{\bfseries Counter-Battery Support}:
Counter-Battery Support costs 14 BSP and can only be used once enemy Artillery Support has been used.
The Target Number for Counter-Battery Support is 7.
Reduce the Target Number by 1 for each use of enemy Artillery Support where a friendly unit had LOS to the point of impact.
Reset the Target Number to 7 if the enemy artillery unit is destroyed by 2 successful Counter-Battery Support attacks.

\subsubsection{Battlefield Support: Assets}

Each side for a scenario must agree to use Battlefield Support: Assets rules.
Commanders may agree to only use Battlefield Support: Assets rules for some types of units, such as conventional infantry units.
Commanders should use the same rules for every unit of the same type in the scenario, BattleTech: Total Warfare or Battlefield Support: Assets.

When converting a unit in a force to an Asset, first compute the Asset skill level by taking the average of the Gunnery and Driving skills, rounding down, and adding 2.
Only BV values for skill 5 and skill 6 are provided on the Asset cards, so unit skill levels may be temporarily degraded for a scenario to a higher skill level.
Use the BV cost of the Asset at the computed skill level.

For BattleTech: Outworlds Wastes, an Asset is considered \emph{damaged} if the Destroy Check Target Number is reduced to (base Destroy Check Target Number + 4) / 2, rounded down.
For example, the Manticore Heavy Tank Asset is considered \emph{damaged} when the Destroy Check Target Number is reduced to (10 + 4) / 2 = 7 while the Warrior H-7 Asset is considered \emph{damaged} when the Destroy Check Target Number is reduced to (5 + 4) / 2 = 4.

If the BV of the Asset at skill 6 is lower than the BV of the unit at Gunnery 4/Driving 5 under Total Warfare rules, scale the \emph{repair} and \emph{replace} costs to account for the unit being less combat effective.
For example, a Manticore Heavy Tank Asset is 420 BV at skill 6, while the Manticore Heavy Tank under Total Warfare rules costs 993 BV.
The Manticore Heavy Tank Asset has 42\% of the BV, so it only costs 279,196 C-bills to \emph{repair} or 558,392 C-bills to \emph{replace} instead of the 660,100 C-bills to \emph{repair} or 1,320,200 C-bills to \emph{replace} the unit using full Total Warfare rules.
Do not apply any modifiers for quirks or DropShip modifications when \emph{repairing} or \emph{replacing} Assets.
