%--------------------------------  Preamble  -----------------------------------

\documentclass[UTF8]{article}

\usepackage[letterpaper, margin=1.25in]{geometry}
\usepackage{graphicx}               % to insert figures
\usepackage{xcolor}                 % colors for e-copies
\usepackage{subcaption}             % subfigures
\usepackage{placeins}               % Float barriers
\usepackage{hyperref}               % PDF hyperreferences
\usepackage{booktabs}
\usepackage{array}

\title{
  Outworld Wastes\\
  ~\\
  \large Casual Battletech League Guide \\
  ~\\
  ~\\
  ~\\
  \includegraphics[width=4in]{../img/Outworlds-Alliance.png}
}
\author{}
\date{}

% Optional PDF information
\ifpdf
\hypersetup{
  pdftitle={Outworld Wastes},
  pdfauthor={Jeremy L Thompson}
}
\fi

%-------------------------------------------------------------------------------
%--------------------------------  Document  -----------------------------------
%-------------------------------------------------------------------------------

\begin{document}

\maketitle

\newpage

Outworlds Wastes is a casual Battletech league framework with simplified logistics rules.
Players will take the role of commander of a mech force searching the Outworld Wastes for lost technology and glory.
Completing objectives in scenarios will earn C-bills that commanders can use to upgrade their forces.
Various forms of Battletech game-play are supported, such as Classic Battletech, Alpha Strike, and Battletech Destiny.\\

\section*{Goals}

\begin{itemize}

\item Have buckets of fun while fostering a friendly and welcoming environment.

\item Give players an opportunity to build personalized lore for their own mechwarrior forces.

\item Provide a lightweight framework for players to track the accomplishments of their forces.

\item Explore Battletech lore and equipment.

\end{itemize}

\newpage

\section{Region Background}

The Outworlds Alliance was founded in 2413 and enjoyed prosperity throughout the Star League Era.
By the start of the Amaris Civil War in 2766, the Outworlds Alliance contained over 135 major systems across 7 administrative districts.
Unfortunately, the Outworlds Alliance suffered during the Succession Wars that followed the fall of the Star League in 2780, and they had to steadily abandon systems they no longer had the resources to support.\\

Clan Snow Raven began exploring the Periphery for resources soon after the battle of Tukayyid ended Operation REVIVAL.
In 3064, Clan Snow Raven and the Outworlds Alliance began developing mutual respect and tentative alliance.
Following their abjuration from the Clan Homeworlds in 3075 as a result of the Wars of Reaving, Clan Snow Raven took refuge in the Outworlds Alliance.
In 3083, Clan Snow Raven and the Outworlds Alliance merged to form the Raven Alliance.\\

By the ilClan Trial in 3151, the Raven Alliance contained only 47 systems.
At least 88 systems that were part of the Outworlds Alliance during the Star League era have been lost.
Many factions are eager to explore these lost worlds in the Outworlds Wastes in search of lost Star League technology or to take refuge from the complex political machinations of Inner Sphere factions.\\

\section{Overview}

You will take the role of commander of a mech force exploring the Outworld Wastes for your faction.
Common factions for the region include

\begin{itemize}

\item Raven Alliance

\item Draconis Combine

\item Federated Suns

\item Mercenary groups

\item Pirate gangs

\item Clan Dark Caste

\end{itemize}

Commanders should pick the faction they are most interested in representing.
While Outworlds Wastes scenarios will generally be focused on the lore of the major factions in the region, listed above, additional factions may have a small presence in the Outworlds Wastes.
For example, the Raven Alliance has relationships with nations on the far side of the Periphery, such as Magistracy of Canopus.\\

Any era after the fall of the Star League can be used for league play.
The era determines unit availability and the most common factions in the Outworld Wastes.\\ 

Commanders will compete with other factions in the Outworlds Wastes to grow their force and recover lost technology.
These scenarios will be primarily designed for Classic Battletech, but content for Alpha Strike and Battletech Destiny is also supported and will be included.\\

\newpage

\section{Force Management}

Unit commanders will start with Battle Value points (BV) budget they can use to purchase their initial units.
Participation in scenarios and accomplishing objectives will earn C-bills for commanders to spend on training their pilots, upgrading units, and acquiring new equipment.

\subsection{Force Construction}

Commanders start with 10,000 BV to acquire initial units for their force.
BV costs for all units are listed in the \href{http://www.masterunitlist.info/}{Master Unit List}.
Force construction must follow these rules:\\

\begin{itemize}

\item Commanders should select units from their era and faction on the \href{http://www.masterunitlist.info/}{Master Unit List}.
Forces can include units with introductory, standard, and advanced technology but should not include experimental units.
For example, the Marauder MAD-3R, Marauder MAD-7R, and Marauder II MAD-6C are legal ilClan era mercenary units while the Marauder II MAD-6M is not.

\item Each force can start with no more than 6,000 BV in mechs.
Commanders are encouraged to use the typical unit composition of their faction but are not required to do so.
For example, inner sphere forces are often organized in lances of 4 mechs while clan forces are organized in stars of 5 mechs and ComStar or Word of Blake forces are organized in Level IIs of 6 mechs.

\item Each force can include any number of supporting units, such as infantry, battle armor, and combat vehicles.

\item Forces cannot contain off-map battlefield support units, such as artillery or aerospace fighters.
However, forces can contain any on-map units.

\item The BV costs of a mech includes the pilot skills.
Pilots may be no better than Gunnery 3/Piloting 4 and no worse than Gunnery 5/Piloting 6.

\item The BV costs of a combat vehicle includes the crew skills.
Crews may be no better than Gunnery 3/Driving 4 and no worse than Gunnery 5/Driving 6.

\item The BV costs of infantry or battle armor includes the troop skills.
Units may be no better than Gunnery 3/Anti-Mech 4 and no worse than Gunnery 5/Anti-Mech 8.

\item Commanders will always have sufficient non-combatant dropship assets to move their force, but these dropships cannot participate in scenarios.

\item Any unspent BV during force creation is lost.

\end{itemize}

One of the goals of the Outworlds Wastes framework is to explore different equipment.
Commanders are encouraged to know where in the rulebooks other commanders can read about the rules pertaining to any special equipment for units in their force.\\

Two sample initial forces are provided; the first force is a Civil War era mercenary company and the second force is an ilClan era Raven Alliance nova.
Mech pilot names are encouraged, as one of the goals is to develop the personalized lore for your force.\\

\newpage

\begin{table}[h!]
\centering
\newcolumntype{R}[1]{>{\raggedleft\let\newline\\\arraybackslash\hspace{0pt}}m{#1}}
\begin{tabular}{| m{12em} m{10em} R{4em} R{4em} R{4em} R{4em} |}
\hline
Battlemech            & Pilot                  & Gunnery & Piloting & BV    & Adjusted BV \\
\hline
Atlas II AS7-D        & Megan 'Meg' Courant    & 3       & 4        & 1,897 & 2,504 \\
Phoenix Hawk PXH-2K   & Dietrich 'Bison' Helge & 4       & 5        & 1,271 & 1,271 \\
Blackjack BJ-2        & Rowan 'Lizard' Baker   & 4       & 5        & 1,148 & 1,148 \\
Locust IIC            & Corey 'Casper' Poole   & 4       & 5        & 1,100 & 1,100 \\
Maxim Hover Transport &                        & 4       & 5        &   764 &   764 \\
Maxim Hover Transport &                        & 4       & 5        &   764 &   764 \\
Galleon GAL-102       &                        & 4       & 5        &   651 &   651 \\
Galleon GAL-102       &                        & 4       & 5        &   651 &   651 \\
Warrior H-7           &                        & 4       & 5        &   295 &   295 \\
Warrior H-7           &                        & 4       & 5        &   295 &   295 \\
IS Std BA, LRR        &                        & 4       & 5        &   255 &   255 \\
IS Std BA, Laser      &                        & 4       & 5        &   231 &   231 \\
\hline
Total                 &                        &         &          &       & 9,929 \\
Unspent               &                        &         &          &       &    71 \\
\bottomrule
\end{tabular}
\caption{Civil War Era Mercenary Force - Meg's Magpies}
\end{table}

\begin{table}[h!]
\centering
\newcolumntype{R}[1]{>{\raggedleft\let\newline\\\arraybackslash\hspace{0pt}}m{#1}}
\begin{tabular}{| m{12em} m{10em} R{4em} R{4em} R{4em} R{4em} |}
\hline
Battlemech             & Pilot           & Gunnery & Piloting & BV    & Adjusted BV \\
\hline
Carrion Crow A         & Sarah Magnus    & 4       & 5        & 1,622 & 1,622 \\
Nova U                 &       Bryn      & 4       & 5        & 1,413 & 1,413 \\
Adder J                &       Ada       & 4       & 5        & 1,222 & 1,222 \\
Kit Fox Prime          &       Soton     & 4       & 5        & 1,085 & 1,085 \\
Fire Moth B            &       Tina      & 4       & 5        &   639 &   639 \\
Gnome BA               &                 & 3       & 4        &   580 &   766 \\
Gnome BA               &                 & 3       & 4        &   580 &   766 \\
Elemental BA, AP Gauss &                 & 3       & 4        &   577 &   762 \\
Elemental BA, AP Gauss &                 & 3       & 4        &   577 &   762 \\
Elemental BA, MicroPL  &                 & 3       & 4        &   480 &   634 \\
Karnov Transport       &                 & 4       & 4        &   152 &   167 \\
Karnov Transport       &                 & 4       & 5        &   152 &   152 \\
\hline
Total                  &                 &         &          &       & 9,990 \\
Unspent                &                 &         &          &       &    10 \\
\hline
\end{tabular}
\caption{ilClan Era Raven Alliance Force - Raven Expeditionary Cluster, Alpha Nova}
\end{table}

\newpage

\subsection{Spending C-Bills}

After earning C-bills from scenarios, commanders can spend C-bills to improve their force.
Possible improvements are listed below.
C-bill costs for all units are listed in the \href{http://www.masterunitlist.info/}{Master Unit List}.\\

\begin{itemize}

\item {\bf Train}: Pay the difference in skill BV modifier times 100,000 C-bills to upgrade a unit's skill levels.
For example, a Gunnery 4/Piloting 5 pilot has a BV modifier of 1.0 and a Gunnery 3/Piloting 4 pilot has a BV modifier of 1.32.
Therefore, it costs 32,000 C-bills to train a 4/5 pilot to be a 3/4 pilot.
Units cannot be upgraded past Gunnery 1/Piloting 2.
Skill BV modifiers can be found on the \href{http://www.masterunitlist.info/}{Master Unit List}.

\item {\bf Replace}: Pay 50\% of the C-bill cost, rounded up, to replace a \emph{destroyed} unit.
If the mech pilot or vehicle crew was killed, the replacement cost includes a 5/6 pilot or crew.
If an entire infantry or battle armor unit was destroyed, the replacement cost includes 5/6 troops.
The new personnel can be trained as above.
See \emph{Battletech: Total Warfare} for the definition of \emph{destroyed} for different types of units.

\item {\bf Repair}: Pay 25\% of the C-bill cost, rounded up, to repair all structure damage and critical components for a mech or combat vehicle that has not been \emph{destroyed}.
If the pilot or crew was killed, the repair cost includes a 5/6 pilot or crew.
Armor damage is repaired for free.

\item {\bf Recruit}: Pay 50\% of the C-bill cost, rounded up, recruit new troops to replace troops in an infantry or battle armor unit that has not been \emph{destroyed}.
For example, if 1 out of 4 troops was killed in a battle armor squad, pay 50\% of the C-bill cost for 1 suit.
To replace 1 troop in a squad of 4 IS Standard Battle Armor with Lasers, pay 293,125 C-bills.
Damage to battle armor troops that survive a scenario is repaired for free.

\item {\bf Refit}: Pay the difference in C-bill cost to refit a unit to a different variant.
A Phoenix Hawk PXH-2 with a 4/5 pilot costs 4,348,840 C-bills.
A Phoenix Hawk PXH-1K with a 4/5 pilot costs 3,628,553.
A commander may pay 720,287 C-bills to convert a PHX-2 into a PHX-1K or to convert a PHX-1K into a PHX-2.

\item {\bf Omni Refit}: Omnimechs can be temporarily converted to a cheaper variant for a scenario for free, but refitting is required to use more expensive variants.
For example, the Carrion Crow C is worth 10,336,492 C-bills.
The Carrion Crow A only costs 9,704,829 C-bills, so a Carrion Crow C can be temporarily configured as a Carrion Crow A for a scenario.
However, a Carrion Crow B costs 15,617,992 C-bills, so a Carrion Crow C would need a 5,281,500 C-bill refit to be converted to the Carrion Crow B variant.
Once the Carrion Crow C is refitted to a Carrion Crow B, the omnimech can be configured as a Carrion Crow A, B, or C for any scenario.

\item {\bf Purchase}: Pay the C-bill cost to get a new unit.
Commanders are encouraged to purchase units from their faction and era list on the \href{http://www.masterunitlist.info/}{Master Unit List}.

\item {\bf Salvage}: Pay 50\% the C-bill cost, rounded up, to salvage units that you destroyed in a scenario.
A War Crow Prime costs 22,057,358 C-bills.
A salvaged War Crow Prime costs 11,028,679 C-bills.
The new pilot of the salvaged unit can be trained as above.
Salvage is the primary way for commanders to get units that are not on their \href{http://www.masterunitlist.info/}{Master Unit List} faction list.

\item {\bf Sell}: Commanders can sell old equipment for 50\% of the C-bill cost, rounded up.
A Locust LCT-1E costs 1,574,200 C-bills and can be sold for 787,100 C-bills.
Commanders can earn 25\% of the C-bill cost for selling a salvaged unit instead of paying 50\% of the C-bill cost to repair the unit.
A salvaged War Crow Prime could be sold to earn 5,514,340 C-bills instead of paying 11,028,679 C-bills to repair it.

\end{itemize}

\newpage

\section{Scenarios}

Commanders earn C-bills to spend on their forces through participation in scenarios and accomplishing objectives.
Common formats for the scenarios include

\begin{itemize}

\item {\bf Classic Battletech}: Scenarios for this format will primarily focus on mid scale combat, with each side controlling approximately one lance with supporting assets.

\item {\bf Alpha Strike}: Scenarios for this format will primarily focus on large scale combat, with each side controlling approximately one company with supporting assets.

\item {\bf Battletech Destiny}: Scenarios for this format will focus on small scale combat, with each side controlling approximately one or two mechs.

\end{itemize}

Scenarios award BV in two ways, through participation and completing objectives.
The BV awarded in a scenario will tend to follow these guidelines

\begin{itemize}

\item {\bf Participation}: Forces earn 5,000 C-bills for every 10 BV for the scenario, with a minimum of 2,500,000 C-bills.
For example, a 6,000 BV vs 6,000 BV scenario will have a base payout of 3,000,000 C-bills.
This C-bill payment represents the baseline cost of a mercenary contract or supplies sent by a faction.

\item {\bf Objectives}: Forces earn 10,000,000 C-bills for completing primary objectives and 5,000,000 C-bills for completing secondary objectives.
This C-bill payment represents bonus pay in a mercenary contract and the value of resources or technology acquired by completing mission objectives.

\end{itemize}

Outworlds Wastes scenarios will often be built to represent lore and objectives relevant to specific worlds in the Outworlds Wastes.
The Outworlds Wastes scenarios will often include special bonuses, such as recovering equipment from the 61st Royal Jump Infantry Division so a commander can add an advanced jump infantry unit to their force for free.\\

\subsection{Scenario Balancing}

One of the goals for the Outworlds Wastes league framework is to foster and friendly and welcoming environment.
It is common for commanders with decades of experience playing Battletech to be participating alongside new people in the hobby.
Here we propose some options to help balance scenarios so gameplay is welcoming to newer commanders while also staying fresh and challenging for more experienced commanders.\\

\begin{itemize}

\item {\bf Setup}: When setting up a scenario, slight preference should generally be given to the commander whose force has the lower total BV, including all units and pilots.
For example, the commander with the lowest total BV could be offered the choice between attacking and defending for the casual scenarios given below.
For a scenario with a terrain setup phase, the commander with the lowest total BV could be offered the first placement of terrain piece.

\item {\bf 2v2}: Many scenarios are described as 1v1; however these scenarios can often support 2v2 or similar play.
When playing on teams, experience should be divided roughly equally between the two teams.
Teammates are encouraged to collaborate on strategy for the scenario.

\end{itemize}

\subsection{Scoring Casual Scenarios}

While there will be Outworlds Wastes themed scenarios, the Outworlds Wastes framework also supports scoring casual games between commanders to give their forces more chances to earn BV and glory.
Some generic scenarios are included here as examples.\\

These scenarios are modified from the scenarios in \emph{Battletech: Total Warfare} and \emph{Battletech: Alpha Strike Commander's Edition}.
See those books for a full description of the scenarios.
The scoring provided here is subject to future balance adjustments.\\

\subsubsection{Classic Battletech}

Five of the six scenario types from \emph{Battletech: Total Warfare} are included here.
The Standup Fight does not fit the narrative nature of Outworlds Wastes.\\

\begin{itemize}

\item {\bf Extraction}: Each side brings the same BV.
The attacker's primary objective is to extract the hidden asset by having a unit carry it to the attacker's home edge.
The defender's primary objective is to steal the hidden asset by having a unit carry it to the defender's home edge.
A secondary objective for all commanders is to cripple or destroy a unit carrying the hidden asset.
The secondary objective can only be awarded once per commander.

\item {\bf Breakthrough}: Each side brings the same BV.
The attacker's primary objective is to reach their home edge, and the defender's primary objective is to prevent attackers from reaching that edge.
10,000,000 C-bills for the primary objective is awarded proportionally, based upon the percentage of the attacker's initial BV that reaches the attacker's home edge.
Crippled units may exit off any map edge, but only only exiting off the attacker's home edge counts towards the attacker meeting the primary objective.
The attacker's secondary objective is to cripple or destroy the defender's forces, and the defender's secondary objective is to preserve their force.
5,000,000 C-bills for the secondary objective is awarded proportionally, as above. 

\item {\bf Chase}: The defending side brings double the BV of the attacking side.
The attacking side has a high value asset that moves as fast as their slowest unit.
The attacker's primary objective is to extract the high value asset by having it cross the attacker's home edge.
The defender's primary objective is to cripple or destroy the high value asset.
The attacker's secondary objective is to reach their home edge, and the defender's secondary objective is to prevent attackers from reaching that edge.
5,000,000 C-bills for the secondary objective is awarded proportionally, based upon the percentage of the attacker's initial BV that reaches the attacker's home edge.

\item {\bf Hide and Seek}: The attacking side brings double the BV of the defending side.
The attacker's primary objective is to cripple or destroy every defending unit.
The defender's primary objective is to cripple or destroy one attacking unit for every defending unit.
Crippled units may exit off any map edge.

\item {\bf Hold the Line}: Each side brings the same BV; attacker should have twice as many units as the defender.
The attacker's primary objective is to cripple or destroy every defending unit.
The defender's primary objective is to cripple or destroy one attacking unit for every defending unit.
Crippled units may exit off any map edge.

\end{itemize}

\newpage

\subsubsection{Alpha Strike}

Four of the five scenario types from \emph{Battletech: Alpha Strike Commander's Edition} are included here.
The Standup Fight does not fit the narrative nature of Outworlds Wastes.\\

\begin{itemize}

\item {\bf Capture the Flag}: Each side brings the same PV and similar numbers of units.
The primary objective of each side is to capture the enemy flag.
The secondary objective of each side is to defend their own flag.

\item {\bf King of the Hill}: Each side brings the same PV and similar numbers of units.
The primary objective of each side is to control the objective zone.
10,000,000 C-bills are awarded proportionally based upon the number of turns the objective is controlled.

\item {\bf Hold the Line}: The defender has 25\% more PV than the attackers.
The attacker's primary objective is to cripple or destroy every defending unit.
The defender's primary objective is to cripple or destroy one attacking unit for every defending unit.
Crippled units may exit off any map edge.
The defender's secondary objective is to hold the line for 3 turn for every 4 defending units.
The attacker's secondary objective is to force all defending units to retreat before their time limit.

\item {\bf Reconnaissance}: Each side brings the same PV and similar numbers of units.
The attacker's primary objective is to locate all hidden objectives.
The defender's primary objective is to prevent the attacker from locating all hidden objectives.
10,000,000 C-bills are awarded proportionally for the number of hidden objectives located by the attackers.

\end{itemize}

\end{document}
%-------------------------------------------------------------------------------