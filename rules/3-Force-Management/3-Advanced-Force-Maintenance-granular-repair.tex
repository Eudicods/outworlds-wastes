\item {\bfseries Granular Repair}: Calculate the repair costs based upon the percentage of the internal structure damaged.
Armor is still repaired for free.
The repair cost for Omni units is still based upon the fielded variant.
A 5/6 pilot or crew is still included in the repair cost.

For BattleTech, calculate \emph{repair} costs as 50\% of the unit's C-bill cost, multiplied by the percentage of internal sections damaged.
The number of internal sections depends upon the unit; for example, a 'Mech has 8 internal sections.
Treat a combat vehicle's motive system as a single additional internal section.
For example, if a ground combat vehicle with a turret has motive system damage and 1 damaged internal section, pay 2 / 6 = 33.3\% of 50\% of the C-bill cost of the unit, or 16.67\% of the C-bill cost of the unit.
It would cost 719,250 C-bills to repair a Demolisher Heavy Tank (Gauss) damaged in this way, instead of 1,078,875 C-bills per the standard \emph{repair} rules.

For Alpha Strike, calculate \emph{repair} costs as 50\% of the unit's C-bill cost, multiplied by the percentage of internal structure bubbles damaged.
For example, if a unit has only 1 out of 3 structure bubbles remaining, pay 2 / 3 = 66.7\% of 50\% of the C-bill cost of the unit, or 33.3\% of the C-bill cost of the unit.
It would cost 3,756,752 C-bills to repair a Wolverine WVR-7D damaged in this way, instead of 2,817,564 C-bills per the standard \emph{repair} rules.
