\emph{BattleTech: Chaos Campaign} provides campaign rules that have lower complexity than \emph{BattleTech: Campaign Operations} and is available as a free PDF.
The Chaos Campaign rules are used for campaigns in \emph{BattleTech: Chaos Campaign: Succession Wars} as well as \emph{BattleTech: Spotlight On} and \emph{BattleTech: Turning Points} publications and \emph{Shrapnel}, the BattleTech magazine.
The \emph{Hot Spots: Hinterlands} sourcebook or the \emph{BattleTech: Mercenaries} box set adds a contract system for the \emph{BattleTech: Chaos Campaign} system.

\emph{BattleTech: Outworlds Wastes} can be used as the logistics system for \emph{BattleTech: Chaos Campaign} with the \emph{BattleTech: Mercenaries} box set extension.

\subsubsection{Contract Terms}

1,000 Supply Points for \emph{Mercenary Chaos Campaign} is worth 3,500,000 C-bills in \emph{BattleTech: Outworlds Wastes}.
A standard 10,000 BV \emph{BattleTech: Outworlds Wastes} force is equivalent to a Scale 2 force in \emph{Mercenary Chaos Campaign}.
These conversion rules assume one scenario per month for the ease of computing logistics costs.

These following contract terms in \emph{Mercenary Chaos Campaign} interact with \emph{BattleTech: Outworlds Wastes} as described below.

\begin{itemize}

\item {\bfseries Base Pay}: \emph{BattleTech: Outworlds Wastes} assumes the employer or supporting faction covers 100\% of the base operating costs for the force.
When using the \emph{Mercenary Chaos Campaign} system, the base operating cost for a \emph{BattleTech: Outworlds Wastes} force is 3,500,000 C-bills.
If the contract provides lower than 100\% Base Pay, the commander must pay these operating costs after each scenario.
For example, if the contract has 80\% Base Pay, the commander pays 700,000 C-bills after each scenario as basic operating costs.
Similarly, if the contract has higher than 100\% base pay, the commander earns extra C-bills after each scenario.

\item {\bfseries Straight Support}: Reduce the \emph{repair} and \emph{recruit} costs by the percentage of Straight Support in the contract.
For example, if the contract offers 80\% Straight Support, the commander only pays 20\% of the \emph{repair} and \emph{replace} costs.

\item {\bfseries Battle Support}: Reduce the \emph{replace} costs by the percentage of Battle Support in the contract.
For example, if the contract offers 10\% Battle Support, the commander only pays 90\% of the \emph{replace} costs.
Note that a contract with Battle Support also grants 100\% Straight Support, so the commander only pays the remaining portion of the \emph{replace} costs and none of the \emph{repair} or \emph{recruit} costs.

\item {\bfseries Salvage Rights}: Only collect the percent of the revenue from \emph{salvage} granted by the Salvage Rights.
For example, if the contract offers 60\% Salvage Rights, then the commander only earns 15\% of the C-bill cost when selling salvaged units instead of the typical 25\% of the C-bill cost.
Also, the Salvage Rights limit how many units the commander may salvage and add to their force.
Multiply the number of enemy units destroyed by the Salvage Rights percentage and round down to determine how many units may be salvaged and added to their force.
A minimum of 1 unit can always be salvaged and added to their force.
For example, if a commander destroys 5 enemy units and has 60\% Salvage Rights, then they may add up to 3 of those units to their force.
If a commander destroys 2 enemy units and has 40\% Salvage Rights, then they may add only 1 of those units to their force.

\item {\bfseries Command Rights}: The specific scenarios determine what effects, if any, Command Rights have on the scenario.

\item {\bfseries Transportation Terms}: If using transportation costs, use the 1,000 Supply Points to 3,500,000 C-bills conversion to compute transportation costs.
The employer or supporting faction covers the negotiated portion of the C-bill cost.
If not tracking transportation costs, then Transportation Terms must be set to 100\% (step 9).

\end{itemize}

League organizers may allow a limited set of contracts or may allow commanders to negotiate contract terms per the rules in the \emph{BattleTech: Mercenaries} box set.
Reputation cannot be used to modify contract terms unless allowed by league organizers.
League organizers set the frequency at which contracts may be renegotiated.
If the renegotiation occurs while the force remains on the same planet, then the Transportation Terms cannot be renegotiated.

\subsubsection{Default Contracts}

Two typical contracts for \emph{BattleTech: Outworlds Wastes} are given here.
As described in the \emph{BattleTech: Mercenaries} box set, commanders may negotiate to adjust the terms of their contract.

The first default contract replicates the \emph{BattleTech: Outworlds Wastes} system.
Commanders have 100\% Base Pay (step 7), Independent Command Rights (step 11), 100\% Salvage Rights (step 13), no Support Rights (step 1), and 100\% Transportation Terms (step 9).
Force maintenance works as described in the \hyperref[subsec:force_maintenance]{Force Maintenance} section.

The second default contract significantly reduces the maintenance cost for commanders while also reducing salvage and transportation payments.
Commanders have 100\% Base Pay (step 7), House Command Rights (step 7), 40\% Salvage Rights (step 7), 10\% Battle Support Rights (step 7), and 50\% Transportation Terms (step 7).
Force maintenance works as described in the \hyperref[subsec:force_maintenance]{Force Maintenance} section with the following modifications.

\begin{description}

\item {\bfseries Replace}: Pay 45\% of the C-bill cost, rounded up, to replace a \emph{destroyed} unit.

\item {\bfseries Repair}: Pay 0\% of the C-bill cost, rounded up, to repair all internal damage and critical components for a unit that has not been \emph{destroyed}.

\item {\bfseries Recruit}: Pay 0\% of the C-bill cost, rounded up, to replace troops in an infantry or Battle Armor unit that was not \emph{destroyed}.

\item {\bfseries Salvage}: Recover enemy units that were \emph{destroyed} in a scenario.
You may add up to 40\% of the destroyed enemy units to your force, rounded down, with a minimum of 1 unit.
Pay 50\% the C-bill cost, rounded up, to add salvaged enemy units to your force.
Also, sell components from any destroyed enemy unit that was not added to your force to earn 10\% of the unit's C-bill cost.

\end{description}

\subsubsection{Scenario Scoring}

Organizers may score scenarios by using the Supply Point payments given in the \emph{Mercenary Chaos Campaign} and using the 1,000 Supply Points to 3,500,000 C-bills conversion.
This scoring only provides three possible payments for the primary objective in a scenario, 50\%, 100\%, or 150\% of the 3,5000,000 base pay for a Scale 2 force.

For a more granular scoring for any \emph{BattleTech: Chaos Campaign} scenario, award the 7,000,000 C-bills payment for the primary objective proportional to the victory points or supply points earned by each side.
For example, if one side earned 8 victory points and the other side earned 13 victory points out of a total 25 possible victory points, then award 2,240,000 C-bills to the first side and 3,640,000 C-bills to the second side.

If the scenario has secondary objectives, the maximum secondary objective payment should be 3,000,000 C-bills.

League organizers may make adjustments to these guidelines as desired.

\subsubsection{Time Between Tracks}

\emph{BattleTech: Outworlds Wastes} replaces the Supply Point (SP) costs for the activities in the \emph{Time Between Tracks} section of \emph{Mercenary Chaos Campaign} with the \emph{BattleTech: Outworlds Wastes} Force Management rules.
The following adjustments apply to the standard \emph{BattleTech: Outworlds Wastes} rules for Force Management.

\begin{itemize}

\item {\bfseries MechWarrior Wounds}: MechWarrior wounds heal at a rate of one box per month.
At the start of each month, remove one hit from each injured MechWarrior.
For example, if a MechWarrior had taken 3 hits, it would take 3 months to fully heal; they would be back to 0 hits at the start of the third month following their injury.

\item {\bfseries Repair Time}: Repairs are completed at the start of the month.
If two or more scenarios are played in the same month, \emph{repairs} and \emph{replacements} from previous scenarios in the month are not completed before the later scenarios.

\item {\bfseries Training Time}: Training is completed at the start of the month.
If two or more scenarios are played in the same month, \emph{training} started after previous scenarios in the month is not completed before the later scenarios.

\item {\bfseries Purchase and Sell Time}: \emph{Purchases} and \emph{sales} apply immediately.
If two or more scenarios are played in the same month, purchase or sales made after earlier scenarios immediately change the unit availability for later scenarios.

\item {\bfseries SPAs}: New SPAs earned by units apply immediately.
If two or more scenarios are played in the same month, SPAs earned after earlier scenarios immediately apply to units for later scenarios.

\end{itemize}

\subsubsection{Scales of Play}

The \emph{BattleTech: Outworlds Wastes} rules correspond to Scale 2 in \emph{Mercenary Chaos Campaign}.
All of the standard and agreed upon optional rules for \emph{BattleTech: Outworlds Wastes} apply at this scale.

To play at Scale 1, use modified \emph{BattleTech: Outworlds Wastes} Event rules.
Your force has a modified Leopard dropship with 4 configurable bays.
Your initial force has up to 3,000 BV in 'Mechs and 2,000 BV in infantry and combat vehicles.
Commanders may use any Battlefield Support: Strikes but may only use Battlefield Support: Assets for units in their force unless league organizers state otherwise.
